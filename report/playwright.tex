\chapter{Playwright}\label{ch_playwright}
Playwright je open-source biblioteka za automatizaciju testiranja web preglednika i web skrapanja koju je razvio Microsoft. Omogućuje automatizaciju testiranja web aplikacija na Chromiumu, Firefoxu i WebKit-u s jednim API-jem.

Prednosti Playwrighta:

\begin{itemize}

\item Jednostavan za korištenje: Playwright ima intuitivan API koji je sličan JQuery-ju i Cypress-u.
\item Brz i pouzdan: Playwright je optimiziran za brzinu i pouzdanost, što ga čini idealnim za testiranje web aplikacija u produkciji.
\item Svestran: Playwright se može koristiti za testiranje različitih tipova web aplikacija, uključujući jednostavne web stranice, jednostruke web aplikacije (SPA) i višestruke web aplikacije (MPA).
\item Podržava više jezika: Playwright se može koristiti s raznim jezicima programiranja, uključujući JavaScript, TypeScript, Python, Java i C\#.
\end{itemize}

Playwright se može koristiti za:
\begin{itemize}
\item Automatizaciju UI testova: Playwright se može koristiti za pisanje automatiziranih UI testova koji provjeravaju funkcionalnost web aplikacija.
\item Web skrapanje: Playwright se može koristiti za prikupljanje podataka sa web stranica.
\item Generiranje screenshot-ova i videozapisa: Playwright se može koristiti za generiranje screenshot-ova i videozapisa web stranica.
\end{itemize}
    
\section{Opis i pregled paketa}
Sistemski zahtjevi za pokretanje Playwrighta su: \footnote{\url{https://playwright.dev/docs/intro\#system-requirements}}
\begin{itemize}
    \item Node.js 18 ili noviji
    \item Windows 10 ili noviji, Windows Server 2016 ili noviji ili Windows Subsystem for Linux (WSL),
    \item MacOS 12 Monterey ili noviji
    \item Debian 11, Debian 12, Ubuntu 20.04 ili Ubuntu 22.04, sa x86-64 ili arm64 arhitekturom.
\end{itemize}

Omogućava testiranje na Chrom Chromium, Firefox i WebKit engenima \footnote{\url{https://www.npmjs.com/package/playwright\#documentation--api-reference}} koji se koriste u modernim web preglednicima.

Paket omogućava izvršavanje testova u UI načinu rada kao i u \emph{headless} načinu rada prilikom kojeg se ne vide koraci kako se kreće po web stranici nego se na kraju testa dobije izvještaj o uspješnosti testiranja.
To je vrlo koristno kada se koriste automatski načini objavljivanja koda koji onda može izvršiti testiranje prilikom svake promjene koda.

\section{Instalacija}

Najjednostavniji način za instalaciju Playwright paketa je putem npm alata koristeći naredbu
\begin{verbatim}
npm init playwright@latest
\end{verbatim}
te će to instalirati paket i pokrenuti postupak inicijalizacije paketa.
Osim \texttt{npm}, može se koristiti i \texttt{yarn} ili \texttt{pnpm}, ovisno o osobnim preferencijama.
Tokom inicijalizacije može se birati nekoliko postavki:
\begin{itemize}
    \item Odabrati TypeScript ili JavaScript (standardno je TypeScript)
    \item Odabrati ime direktorija koji će sadržavati testove (standardno je 'test' ili 'e2e' - end to end, ako 'test' već postoji)
    \item Dodati GitHub Action workflow za automatsko izvršavanje testova prilikom objave izvornog koda na GitHub servisu
    \item Instalirati potrebne preglednike koji će se koristiti za testiranje
    
\end{itemize}

Playwright će nakon toga kreirati potrebne direktorije i datoteke za konfiguraciju kao i primjer jednog testa za lakši početak
\begin{verbatim}
playwright.config.ts
package.json
package-lock.json
tests/
  example.spec.ts
tests-examples/
  demo-todo-app.spec.ts
\end{verbatim}

Primjetimo kako je uvrijeđena norma da se datoteke koje sadrže testove imaju \texttt{.spec} ispred oznake tipa datoteke uz zadržavanje istog imena. Čak ih i razni editori koda označavaju s drugim ikonama kako bi bili vizualno lakše raspoznatljivi od datoteka koje sadrže izvorni komponenti kao što je vidljivo na slici \ref{img:filesLogos}.
\begin{figure}[!h]\begin{center}
    \includegraphics[width=1\textwidth]{"img/filesLogos"}
    \caption{Izgled ikona sa izvornim kodom i testom za komponentu}\label{img:filesLogos}
\end{center}\end{figure}

Ukoliko se inicijalizacija vrši unutar već postojećeg projekta, što je najčešće i slučaj, konfiguracija zavisnih paketa će biti dodana u postojeću \texttt{package.json} datoteku.

\begin{figure}[!h]\begin{center}
    \includegraphics[width=1\textwidth]{"img/pwInit"}
    \caption{Ekran nakon uspješne instalacije i inicijalizacije Playwright paketa}\label{img:pwInit}
\end{center}\end{figure}

Na slici \ref{img:pwInit} vidimo kako izgleda uspješna instalacija i inicijalizacija Playwright paketa.

\texttt{playwright.config.ts} datoteka sadrži konfiguracije testova kao npr 
\begin{itemize}
    \item koji se preglednik koristi, 
    \item koja je veličina prozora preglednika, 
    \item koji se mobilni uređaj koristi u slučaju tesitiranja na mobilnim preglednicima,
    \item standardno očekivano vrijeme ispunjenja testa (timeout)
\end{itemize}
te mnogi drugi preddefinirane i prilagođene opcije konfiguracije.


\section{Osnovni test}

\section{Kontinuirana integracija - CI}
