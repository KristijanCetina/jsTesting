\chapter{Playwright}\label{ch_playwright}
Playwright je open-source biblioteka za automatizaciju testiranja web preglednika i web skrapanja koju je razvio Microsoft. Omogućuje automatizaciju testiranja web aplikacija na Chromiumu, Firefoxu i WebKit-u s jednim API-jem.

Prednosti Playwrighta:

\begin{itemize}

\item Jednostavan za korištenje: Playwright ima intuitivan API koji je sličan JQuery-ju i Cypress-u.
\item Brz i pouzdan: Playwright je optimiziran za brzinu i pouzdanost, što ga čini idealnim za testiranje web aplikacija u produkciji.
\item Svestran: Playwright se može koristiti za testiranje različitih tipova web aplikacija, uključujući jednostavne web stranice, jednostruke web aplikacije (SPA) i višestruke web aplikacije (MPA).
\item Podržava više jezika: Playwright se može koristiti s raznim jezicima programiranja, uključujući JavaScript, TypeScript, Python, Java i C\#.
\end{itemize}

Playwright se može koristiti za:
\begin{itemize}
\item Automatizaciju UI testova: Playwright se može koristiti za pisanje automatiziranih UI testova koji provjeravaju funkcionalnost web aplikacija.
\item Web skrapanje: Playwright se može koristiti za prikupljanje podataka sa web stranica.
\item Generiranje screenshot-ova i videozapisa: Playwright se može koristiti za generiranje screenshot-ova i videozapisa web stranica.
\end{itemize}
    
\section{Opis i pregled paketa}
Sistemski zahtjevi za pokretanje Playwrighta su: \footnote{\url{https://playwright.dev/docs/intro\#system-requirements}}
\begin{itemize}
    \item Node.js 18 ili noviji
    \item Windows 10 ili noviji, Windows Server 2016 ili noviji ili Windows Subsystem for Linux (WSL),
    \item MacOS 12 Monterey ili noviji
    \item Debian 11, Debian 12, Ubuntu 20.04 ili Ubuntu 22.04, sa x86-64 ili arm64 arhitekturom.
\end{itemize}

Omogućava testiranje na Chromium, Firefox i WebKit engenima \footnote{\url{https://www.npmjs.com/package/playwright\#documentation--api-reference}} koji se koriste u modernim web preglednicima.

Paket omogućava izvršavanje testova u UI načinu rada kao i u \emph{headless} načinu rada prilikom kojeg se ne vide koraci kako se kreće po web stranici nego se na kraju testa dobije izvještaj o uspješnosti testiranja.
To je vrlo koristno kada se koriste automatski načini objavljivanja koda koji onda može izvršiti testiranje prilikom svake promjene koda.

\section{Instalacija}

Najjednostavniji način za instalaciju Playwright paketa je putem npm alata koristeći naredbu
\begin{verbatim}
npm init playwright@latest
\end{verbatim}
te će to instalirati paket i pokrenuti postupak inicijalizacije paketa.
Osim \texttt{npm}, može se koristiti i \texttt{yarn} ili \texttt{pnpm}, ovisno o osobnim preferencijama.
Tokom inicijalizacije može se birati nekoliko postavki:
\begin{itemize}
    \item Odabrati TypeScript ili JavaScript (standardno je TypeScript)
    \item Odabrati ime direktorija koji će sadržavati testove (standardno je 'test' ili 'e2e' - end to end, ako 'test' već postoji)
    \item Dodati GitHub Action workflow za automatsko izvršavanje testova prilikom objave izvornog koda na GitHub servisu
    \item Instalirati potrebne preglednike koji će se koristiti za testiranje
    
\end{itemize}

Playwright će nakon toga kreirati potrebne direktorije i datoteke za konfiguraciju kao i primjer jednog testa za lakši početak
\begin{verbatim}
playwright.config.ts
package.json
package-lock.json
tests/
  example.spec.ts
tests-examples/
  demo-todo-app.spec.ts
\end{verbatim}

Primjetimo kako je uvrijeđena norma da se datoteke koje sadrže testove imaju \texttt{.spec} ispred oznake tipa datoteke uz zadržavanje istog imena. Čak ih i razni editori koda označavaju s drugim ikonama kako bi bili vizualno lakše raspoznatljivi od datoteka koje sadrže izvorni komponenti kao što je vidljivo na slici \ref{img:filesLogos}.
\begin{figure}[!h]\begin{center}
    \includegraphics[width=1\textwidth]{"img/filesLogos"}
    \caption{Izgled ikona sa izvornim kodom i testom za komponentu}\label{img:filesLogos}
\end{center}\end{figure}

Ukoliko se inicijalizacija vrši unutar već postojećeg projekta, što je najčešće i slučaj, konfiguracija zavisnih paketa će biti dodana u postojeću \texttt{package.json} datoteku.

\texttt{playwright.config.ts} datoteka sadrži konfiguracije testova kao npr 
\begin{itemize}
    \item koji se preglednik koristi, 
    \item koja je veličina prozora preglednika, 
    \item koji se mobilni uređaj koristi u slučaju tesitiranja na mobilnim preglednicima,
    \item standardno očekivano vrijeme ispunjenja testa (timeout)
\end{itemize}
te mnogi drugi preddefinirane i prilagođene opcije konfiguracije.

Na slici \ref{img:pwInit} vidimo kako izgleda uspješna instalacija i inicijalizacija Playwright paketa.
\begin{figure}[!h]\begin{center}
    \includegraphics[width=1\textwidth]{"img/pwInit"}
    \caption{Ekran nakon uspješne instalacije i inicijalizacije Playwright paketa}\label{img:pwInit}
\end{center}\end{figure}
\section{Generiranje testova}
Playwright ima mogućnost generiranja testova tako što snima klikanje miša korisnika te to prevodi u kod koji se može izvršavati.
To je vrlo jednostavan i praktičan način da čak i totalni početnici mogu krenuti s izradom automatskih testova, a kasnije s iskustvom se ti testovi mogu rafinirati i poboljšavati.
Često je i tako generirai kod dovoljno dobar za upotrebu kod jednostvanijih slučajeva, a zasigurno je dobar za brzo sastavljanje testova da se izbjegne često dosatno tipkanje djelova koda koji se neće kasnije ponovno upotrebljavati.
To uklanja potrebu za pisanje prilagođenih funkcija kao što radimo prilikom izrade projekta koji planiramo dugo vremena održavati i na taj način se može uštediti dosta vremena.
Iako, treba biti oprezan s time jer često ušteda vremena na početku vodi do puno utrošenog vremena kasnije, ali to je tema koja je izvan okvira ovog rada.
\section{Pokretanje alata za generiranje testova}

Alat za generiranje testova se pokreće putem \texttt{codegen} naredbe koja prima argument URL web stranice za koju se želi generirati testovi.
URL nije obavezan i može se pokrenuti alat bez njega, a zatim dodati URL izravno u prozoru preglednika.

Pokazati ćemo to na promjeru koji se nalazi u službenoj dokumentaciji \footnote{\url{https://playwright.dev/docs/codegen-intro\#running-codegen}} koristeći naredbu
\begin{verbatim}
npx playwright codegen demo.playwright.dev/todomvc
\end{verbatim}
\begin{figure}[!h]\begin{center}
    \includegraphics[width=1\textwidth]{"img/codegenInterface"}
    \caption{Izgled sučelja za generiranje testova}\label{img:pwCodeGen}
\end{center}\end{figure}

Na slici \ref{img:pwCodeGen} je prikazan izgled sučelja za generiranje testova koji se sastoji od nekoliko djelova:
\begin{itemize}
    \item prozor preglednika unutar kojeg se izvršava aplikacija - označen s crvenim okvirom (gore)
    \item prozor unutar kojeg se prikazuje generirani kod - označen s žutim okvirom (dolje)
    \item Lokator koji će Playwright koristiti se prikazuje kada se stavi miš preko elementa - označen sa zelenim okvirom (unutar prozora preglednika)
\end{itemize}


\section{Kontinuirana integracija i testiranje}\label{CI/CD}
Uvođenje kontinuirane integracije (CI) i kontinuirane dostave (CD) predstavlja jedan od ključnih elemenata suvremenog razvoja softvera, omogućujući timovima bržu, pouzdaniju i konzistentniju isporuku softverskih rješenja. 
U nastavku će biti opisano kako i zašto napraviti jednostavan CI/CD pipeline-a za automatizirano testiranje pomoću Playwrighta.

CI/CD pipeline je automatizirani niz koraka koji omogućuje brzu i pouzdanu isporuku aplikacija. CI/CD pipeline se sastoji od niza automatiziranih procesa koji uključuju izgradnju (build), testiranje i distribuciju (deployment) aplikacije. 
Kroz automatizaciju ovih koraka, CI/CD pipeline pomaže u smanjenju rizika od grešaka, ubrzava proces isporuke te osigurava dosljednost i kvalitetu softverskih rješenja.

% \subsection{Postavljanje Playwrighta u CI/CD pipeline}

Integracija Playwright-a u CI/CD pipeline omogućava automatizirano izvođenje testova pri svakoj promjeni koda, čime se osigurava da sve funkcionalnosti aplikacije rade ispravno prije nego što se promjene implementiraju u produkciju.

Definicija CI/CD Pipeline-a: Nakon konfiguracije repozitorija, potrebno je definirati CI/CD pipeline. To se obično radi pomoću YAML datoteka koje sadrže instrukcije za izgradnju, testiranje i distribuciju aplikacije. U nastavku je primjer osnovne konfiguracije za GitHub Actions:

\begin{verbatim}
name: Playwright Tests

on:
  push:
    branches:
      - main
  pull_request:
    branches:
      - main

jobs:
  test:
    runs-on: ubuntu-latest

    steps:
    - name: Checkout code
      uses: actions/checkout@v2

    - name: Setup Node.js
      uses: actions/setup-node@v2
      with:
        node-version: '14'

    - name: Install dependencies
      run: npm install

    - name: Run Playwright tests
      run: npx playwright test

\end{verbatim}
Ova konfiguracija definira akcije koje će se pokrenuti pri svakom push-u ili pull request-u na glavnu granu repozitorija. Akcije uključuju preuzimanje koda, postavljanje Node.js okruženja, instalaciju zavisnosti i pokretanje Playwright testova.

Nakon definiranja CI/CD pipeline-a, svaki push ili pull request pokreće automatiziranu izgradnju i testiranje aplikacije. 
Playwright testovi se izvršavaju unutar pipeline-a, čime se osigurava da sve promjene koda ne narušavaju postojeću funkcionalnost.

Posljednji korak CI/CD pipeline-a je distribucija aplikacije. 
Ako svi testovi prođu uspješno, aplikacija se automatski distribuira na produkcijsko okruženje. 
Ovaj korak može uključivati različite metode distribucije, kao što su deployment na cloud platforme, generiranje Docker datoteke (image) ili distribucija na serverske klastere.

\subsection*{Prednosti CI/CD Pipelinea za Playwright}
Implementacija CI/CD pipelinea za Playwright donosi brojne prednosti:
\begin{itemize}
    \item Automatizacija: CI/CD pipeline automatizira proces testiranja i distribucije, čime se smanjuje potreba za ručnim intervencijama i povećava produktivnost tima.
    \item Konzistentnost: Automatizirano testiranje osigurava dosljednost i kvalitetu aplikacije, jer se testovi izvršavaju pri svakoj promjeni koda.
    \item Brža isporuka: CI/CD pipeline ubrzava proces isporuke, omogućujući brže uvođenje novih funkcionalnosti i popravaka grešaka.
    \item Rano otkrivanje problema: Redovito izvršavanje testova omogućava rano otkrivanje problema, čime se smanjuje rizik od grešaka u produkcijskom okruženju.
\end{itemize}