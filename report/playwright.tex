\chapter{Playwright}\label{ch_playwright}
Playwright je open-source biblioteka za automatizaciju testiranja web preglednika i web skrapanja koju je razvio Microsoft. Omogućuje automatizaciju testiranja web aplikacija na Chromiumu, Firefoxu i WebKit-u s jednim API-jem.

Prednosti Playwrighta:

\begin{itemize}

\item Jednostavan za korištenje: Playwright ima intuitivan API koji je sličan JQuery-ju i Cypress-u.
\item Brz i pouzdan: Playwright je optimiziran za brzinu i pouzdanost, što ga čini idealnim za testiranje web aplikacija u produkciji.
\item Svestran: Playwright se može koristiti za testiranje različitih tipova web aplikacija, uključujući jednostavne web stranice, jednostruke web aplikacije (SPA) i višestruke web aplikacije (MPA).
\item Podržava više jezika: Playwright se može koristiti s raznim jezicima programiranja, uključujući JavaScript, TypeScript, Python, Java i C\#.
\end{itemize}

Playwright se može koristiti za:
\begin{itemize}
\item Automatizaciju UI testova: Playwright se može koristiti za pisanje automatiziranih UI testova koji provjeravaju funkcionalnost web aplikacija.
\item Web skrapanje: Playwright se može koristiti za prikupljanje podataka sa web stranica.
\item Generiranje screenshot-ova i videozapisa: Playwright se može koristiti za generiranje screenshot-ova i videozapisa web stranica.
\end{itemize}
    
\section{Opis i pregled paketa}
Sistemski zahtjevi za pokretanje Playwrighta su: \footnote{\url{https://playwright.dev/docs/intro\#system-requirements}}
\begin{itemize}
    \item Node.js 18 ili noviji
    \item Windows 10 ili noviji, Windows Server 2016 ili noviji ili Windows Subsystem for Linux (WSL),
    \item MacOS 12 Monterey ili noviji
    \item Debian 11, Debian 12, Ubuntu 20.04 ili Ubuntu 22.04, sa x86-64 ili arm64 arhitekturom.
\end{itemize}

Omogućava testiranje na Chromium, Firefox i WebKit engenima \footnote{\url{https://www.npmjs.com/package/playwright\#documentation--api-reference}} koji se koriste u modernim web preglednicima.

Paket omogućava izvršavanje testova u UI načinu rada kao i u \emph{headless} načinu rada prilikom kojeg se ne vide koraci kako se kreće po web stranici nego se na kraju testa dobije izvještaj o uspješnosti testiranja.
To je vrlo koristno kada se koriste automatski načini objavljivanja koda koji onda može izvršiti testiranje prilikom svake promjene koda.

\section{Instalacija}

Najjednostavniji način za instalaciju Playwright paketa je putem npm alata koristeći naredbu
\begin{verbatim}
npm init playwright@latest
\end{verbatim}
te će to instalirati paket i pokrenuti postupak inicijalizacije paketa.
Osim \texttt{npm}, može se koristiti i \texttt{yarn} ili \texttt{pnpm}, ovisno o osobnim preferencijama.
Tokom inicijalizacije može se birati nekoliko postavki:
\begin{itemize}
    \item Odabrati TypeScript ili JavaScript (standardno je TypeScript)
    \item Odabrati ime direktorija koji će sadržavati testove (standardno je 'test' ili 'e2e' - end to end, ako 'test' već postoji)
    \item Dodati GitHub Action workflow za automatsko izvršavanje testova prilikom objave izvornog koda na GitHub servisu
    \item Instalirati potrebne preglednike koji će se koristiti za testiranje
    
\end{itemize}

Playwright će nakon toga kreirati potrebne direktorije i datoteke za konfiguraciju kao i primjer jednog testa za lakši početak
\begin{verbatim}
playwright.config.ts
package.json
package-lock.json
tests/
  example.spec.ts
tests-examples/
  demo-todo-app.spec.ts
\end{verbatim}

Primjetimo kako je uvrijeđena norma da se datoteke koje sadrže testove imaju \texttt{.spec} ispred oznake tipa datoteke uz zadržavanje istog imena. Čak ih i razni editori koda označavaju s drugim ikonama kako bi bili vizualno lakše raspoznatljivi od datoteka koje sadrže izvorni komponenti kao što je vidljivo na slici \ref{img:filesLogos}.
\begin{figure}[!h]\begin{center}
    \includegraphics[width=1\textwidth]{"img/filesLogos"}
    \caption{Izgled ikona sa izvornim kodom i testom za komponentu}\label{img:filesLogos}
\end{center}\end{figure}

Ukoliko se inicijalizacija vrši unutar već postojećeg projekta, što je najčešće i slučaj, konfiguracija zavisnih paketa će biti dodana u postojeću \texttt{package.json} datoteku.

\texttt{playwright.config.ts} datoteka sadrži konfiguracije testova kao npr 
\begin{itemize}
    \item koji se preglednik koristi, 
    \item koja je veličina prozora preglednika, 
    \item koji se mobilni uređaj koristi u slučaju tesitiranja na mobilnim preglednicima,
    \item standardno očekivano vrijeme ispunjenja testa (timeout)
\end{itemize}
te mnogi drugi preddefinirane i prilagođene opcije konfiguracije.

Na slici \ref{img:pwInit} vidimo kako izgleda uspješna instalacija i inicijalizacija Playwright paketa.
\begin{figure}[!h]\begin{center}
    \includegraphics[width=1\textwidth]{"img/pwInit"}
    \caption{Ekran nakon uspješne instalacije i inicijalizacije Playwright paketa}\label{img:pwInit}
\end{center}\end{figure}

\section{Lokatori}

Lokatori su ključni element u radu s pronalaženjem elemenata nad kojima želimo izvršiti neku radnju.
Ukratko, lokatori predstavljaju način pronalaženja elemenata na stranici u bilo kojem trenutku.
Kroz različite metode, Playwright omogućuje precizno i učinkovito lociranje elemenata, čime se omogućava pouzdanost testova.

Ovo su preporučeni ugrađeni lokatori:

\begin{itemize}
  \item \texttt{page.getByRole()} za lociranje prema eksplicitnim i implicitnim atributima pristupačnosti.
  \item \texttt{page.getByText()} za lociranje prema tekstualnom sadržaju.
  \item \texttt{page.getByLabel()} za lociranje kontrola obrasca prema tekstu pridružene oznake.
  \item \texttt{page.getByPlaceholder()} za lociranje unosa prema rezerviranom mjestu.
  \item \texttt{page.getByAltText()} za lociranje elemenata, obično slika, prema njihovom alternativnom tekstu.
  \item \texttt{page.getByTitle()} za lociranje elemenata prema atributu naslova.
  \item \texttt{page.getByTestId()} za lociranje elemenata na temelju atributa data-testid (moguće je konfigurirati i druge atribute).
\end{itemize}

Primjer korištenja:

\begin{verbatim}
await page.getByLabel('User Name').fill('John');
await page.getByLabel('Password').fill('secret-password');
await page.getByRole('button', { name: 'Sign in' }).click();
await expect(page.getByText('Welcome, John!')).toBeVisible();
\end{verbatim}

Playwright dolazi s više ugrađenih lokatora. Kako bi testovi bili otporniji, preporučuje se prioritiziranje atributa usmjerenih prema korisniku i eksplicitnih ugovora, poput \texttt{page.getByRole()}.

Na primjer, za sljedeću DOM strukturu:

\begin{verbatim}
<button>Sign in</button>
\end{verbatim}

Element se može locirati prema njegovoj ulozi \texttt{button} s nazivom "Sign in":

\begin{verbatim}
await page.getByRole('button', { name: 'Sign in' }).click();
\end{verbatim}

\subsection*{Lociranje prema ulozi}

Lokator \texttt{page.getByRole()} odražava kako korisnici i asistivne tehnologije percipiraju stranicu, primjerice je li neki element gumb ili potvrdni okvir.
Kada se locira prema ulozi, obično treba navesti i dostupno ime kako bi lokator točno odredio element.

Primjer DOM strukture:

\begin{verbatim}
<h3>Sign up</h3>
<label>
  <input type="checkbox" /> Subscribe
</label>
<br/>
<button>Submit</button>
\end{verbatim}

Elementi se mogu locirati prema implicitnim ulogama:

\begin{verbatim}
await expect(page.getByRole('heading', { name: 'Sign up' })).toBeVisible();
await page.getByRole('checkbox', { name: 'Subscribe' }).check();
await page.getByRole('button', { name: /submit/i }).click();
\end{verbatim}

\subsection*{Lociranje prema oznaci}

Većina kontrola obrasca ima pridružene oznake koje se mogu koristiti za interakciju s obrascem.
Ova metoda se koristi kada lociramo polja obrasca:

Primjer DOM strukture:
\begin{verbatim}
<label>Password <input type="password" /></label>
\end{verbatim}

Unos se može ispuniti nakon lociranja prema tekstu oznake:

\begin{verbatim}
await page.getByLabel('Password').fill('secret');
\end{verbatim}

\subsection*{Lociranje prema rezerviranom mjestu (placeholder)}

Unosi mogu imati placeholder atribut koji korisnicima sugerira koji bi se vrijednosti trebali unijeti.
Ova metoda se koristi kada lociramo elemente obrasca koji nemaju oznake, ali imaju tekstove rezerviranog mjesta:

Primjer DOM strukture:

\begin{verbatim}
<input type="email" placeholder="name@example.com" />
\end{verbatim}

Unos se može ispuniti nakon lociranja prema tekstu rezerviranog mjesta:

\begin{verbatim}
await page.getByPlaceholder('name@example.com').fill('playwright@microsoft.com');
\end{verbatim}

\subsection*{Lociranje prema tekstu}

Lokatori prema tekstu omogućuju pronalaženje elementa prema tekstu koji sadrži.
Možete koristiti podstring, točan niz ili regularni izraz.

Primjer DOM strukture:

\begin{verbatim}
<span>Welcome, John</span>
\end{verbatim}

Element se može locirati prema tekstu koji sadrži:

\begin{verbatim}
await expect(page.getByText('Welcome, John')).toBeVisible();
\end{verbatim}

Za točno podudaranje:

\begin{verbatim}
await expect(page.getByText('Welcome, John', { exact: true })).toBeVisible();
\end{verbatim}

Za podudaranje s regularnim izrazom:

\begin{verbatim}
await expect(page.getByText(/welcome, [A-Za-z]+$/i)).toBeVisible();
\end{verbatim}

\subsection*{Lociranje prema alternativnom tekstu}

Sve slike trebale bi imati atribut alt koji opisuje sliku.
Ova metoda se koristi za lociranje slika prema alternativnom tekstu:

Primjer DOM strukture:

\begin{verbatim}
<img alt="playwright logo" src="/img/playwright-logo.svg" width="100" />
\end{verbatim}

Slika se može kliknuti nakon lociranja prema alternativnom tekstu:

\begin{verbatim}
await page.getByAltText('playwright logo').click();
\end{verbatim}

\subsection*{Lociranje prema naslovu}

Elementi se mogu locirati prema odgovarajućem atributu naslova koristeći **page.getByTitle()**.

Primjer DOM strukture:

\begin{verbatim}
<span title='Issues count'>25 issues</span>
\end{verbatim}

Broj problema može se provjeriti nakon lociranja prema tekstu naslova:

\begin{verbatim}
await expect(page.getByTitle('Issues count')).toHaveText('25 issues');
\end{verbatim}

\subsection*{Lociranje prema test id-u}

Testiranje prema test id-evima je najotporniji način testiranja jer će test proći čak i ako se tekst ili uloga atributa promijene.
No, testiranje prema test id-evima nije usmjereno prema korisniku.
Ako je vrijednost uloge ili teksta važna, valja razmotriti korištenje lokatora usmjerenih prema korisniku poput \texttt{page.getByRole()} ili \texttt{page.getByText()}.

Primjer DOM strukture:

\begin{verbatim}
<button data-testid="directions">Itinéraire</button>
\end{verbatim}

Element se može locirati prema njegovom test id-u:

\begin{verbatim}
await page.getByTestId('directions').click();
\end{verbatim}

\subsection*{Podešavanje prilagođenog atributa test id-a}

Prema zadanim postavkama, \texttt{page.getByTestId()} će locirati elemente na temelju atributa \texttt{data-testid}, ili može biti drukčije konfigurirati u testnoj okolini ili pak pozivom \texttt{selectors.setTestIdAttribute()}.

Primjer konfiguracije:

\begin{verbatim}
import { defineConfig } from '@playwright/test';

export default defineConfig({
  use: {
    testIdAttribute: 'data-pw'
  }
});
\end{verbatim}

Te se onda može koristiti dati atribut za lociranje elemenata:

\begin{verbatim}
\begin{verbatim}
<button data-pw="directions">Itinéraire</button>
\end{verbatim}

Element se može locirati kao i obično:

\begin{verbatim}
await page.getByTestId('directions').click();
\end{verbatim}

\subsection*{Lociranje prema CSS-u ili XPath-u}

Ako je apsolutno nužno koristiti CSS ili XPath lokatore, može se koristiti \texttt{page.locator()} za kreiranje lokatora koji uzima selektor opisujući kako pronaći element na stranici.
Playwright podržava CSS i XPath selektore te ih automatski prepoznaje ako izostavite prefiks \texttt{css=} ili \texttt{xpath=}.

Primjeri korištenja:

\begin{verbatim}
await page.locator('css=button').click();
await page.locator('xpath=//button').click();

await page.locator('button').click();
await page.locator('//button').click();
\end{verbatim}

CSS i XPath selektori mogu biti vezani za strukturu DOM-a ili implementaciju, što znači da se mogu prekinuti kada se struktura DOM-a promijeni.
Dugi CSS ili XPath lanci su primjer loše prakse koja vodi do nestabilnih testova:

\begin{verbatim}
await page.locator('#tsf > div:nth-child(2) > div.A8SBwf > div.RNNXgb > div > div.a4bIc > input').click();
await page.locator('//*[@id="tsf"]/div[2]/div[1]/div[1]/div/div[2]/input').click();
\end{verbatim}

Preporuka je izbjegavati CSS i XPath lokatore kad god je to moguće i umjesto toga koristiti lokatore koji su bliži percepciji korisnika stranice, poput \texttt{page.getByRole()} ili definirati eksplicitne testne atribute koristeći test id-ove.

\subsection*{Lociranje u Shadow DOM-u}

Svi lokatori u Playwrightu prema zadanim postavkama rade s elementima u Shadow DOM-u, s iznimkom:
\begin{itemize}
\item Lociranje prema XPath-u ne prolazi kroz shadow root.
\item Shadow root u zatvorenom načinu rada nije podržan.
\end{itemize}

Primjer s prilagođenom web komponentom:

\begin{verbatim}
<x-details role="button" aria-expanded="true" aria-controls="inner-details">
  <div>Title</div>
  #shadow-root
  <div id="inner-details">
    <button>Submit</button>
  </div>
</x-details>
\end{verbatim}

Elementi u Shadow DOM-u mogu se locirati prema njihovim ulogama:

\begin{verbatim}
await page.getByRole('button', { name: 'Submit' }).click();
\end{verbatim}

Elementi u zatvorenom Shadow rootu nisu dostupni lokatorima:

\begin{verbatim}
#shadow-root(mode=closed)
  <button>Submit</button>
\end{verbatim}

\subsection*{Metode lokatora}

Lokator je centar svih akcija i asertacija u Playwrightu:

\begin{itemize}
 
  \item \texttt{locator.click()} klikne na element.
  \item \texttt{locator.fill(value)} ispunjava unos.
  \item \texttt{locator.count()} vraća broj elemenata.
\end{itemize}

Ponašanje lokatora koristi se za automatsko čekanje elemenata.
Playwrightova \texttt{expect(locator).toBeVisible()} metoda čeka dok element ne postane vidljiv prije nego što nastavi.
Lokatori su također robusniji prema nestalnosti DOM-a, čekajući automatski da elementi postanu dostupni i ponovno pokušavajući u slučaju grešaka u mreži ili spore dinamike stranice.

Primjeri:

\begin{verbatim}
await page.locator('text=Submit').click();
await page.locator('input[type="password"]').fill('password');
await expect(page.locator('button')).toHaveCount(3);
\end{verbatim}

Više metoda za rad s lokatorima:
\begin{itemize}
  
  \item  \texttt{locator.first()} locira prvi element.
  \item  \texttt{locator.last()} locira zadnji element.
  \item  \texttt{locator.nth(index)} locira n-ti element.
\end{itemize}

Primjeri:

\begin{verbatim}
await page.locator('button').first().click();
await page.locator('button').last().click();
await page.locator('button').nth(2).click();
\end{verbatim}

\subsection*{Testiranje dinamike}

Lokatori u Playwrightu automatski čekaju da element postane dostupan i ponovo pokušavaju ako se element mijenja, što ih čini otpornima na dinamiku stranica.
Ovdje su neki primjeri kako se to koristi:

Primjeri:

\begin{verbatim}
// Čeka da gumb postane vidljiv i klikne ga
await page.locator('button').click();

// Ispunjava unos kada postane dostupan
await page.locator('input[type="password"]').fill('password');

// Čeka da gumb postane nevidljiv
await expect(page.locator('button')).not.toBeVisible();
\end{verbatim}

Playwrightova \texttt{expect} metoda može se koristiti za različite asertacije, što dodatno povećava fleksibilnost i robusnost testova.

\subsection*{Primjeri naprednog lociranja}

Lokatori mogu biti kombinirani za složenije scenarije:

\texttt{Chaining locators}: Kombinacija više lokatora za precizno ciljanje elementa.

\begin{verbatim}
await page.locator('form').locator('input[name="username"]').fill('john_doe');
\end{verbatim}

\texttt{Filtering locators}: Korištenje filtera kao što su \texttt{has} i \texttt{hasText} za daljnje sužavanje rezultata.

\begin{verbatim}
await page.locator('div').locator('text=Submit').click();
await page.locator('div', { hasText: 'Important' }).click();
\end{verbatim}

\texttt{Working with child locators}: Ciljanje potomaka unutar određenog elementa.

\begin{verbatim}
await page.locator('ul').locator('li >> text=Item 2').click();
\end{verbatim}

\texttt{Relative locators}: Korištenje relativnih lokatora za ciljanje elemenata u odnosu na druge elemente.

\begin{verbatim}
await page.locator('text=Name').locator('..').locator('input').fill('Jane Doe');
\end{verbatim}

\subsection*{Složeniji scenariji lociranja}

Za složenije scenarije, Playwright nudi dodatne mogućnosti:
Korištenje \texttt{nth()} se kada postoji potreba za ciljanjem određenog pojavljivanja elementa među više istovrsnih elemenata.

\begin{verbatim}
await page.locator('button').nth(2).click(); // Klik na treći gumb u nizu
\end{verbatim}

Kombinacija više lokatora za precizno ciljanje.
\begin{verbatim}
await page.locator('section').locator('button', { hasText: 'Submit' }).click();
\end{verbatim}

Rad s elementima unutar Shadow DOM-a.
\begin{verbatim}
await page.locator('my-component').locator('button', { hasText: 'Submit' }).click();
\end{verbatim}

\subsection*{Praktične primjene}

Evo nekoliko praktičnih primjera kako se lokatori koriste u stvarnim scenarijima testiranja:

\subsubsection*{Automatsko popunjavanje obrazaca:}

\begin{verbatim}
await page.getByLabel('Username').fill('test_user');
await page.getByLabel('Password').fill('password123');
await page.getByRole('button', { name: 'Login' }).click();
\end{verbatim}

\subsubsection*{Validacija sadržaja}:

\begin{verbatim}
await expect(page.getByText('Welcome, test_user!')).toBeVisible();
\end{verbatim}

\subsubsection*{Interakcija s pop-up prozorima}:

\begin{verbatim}
await page.getByRole('button', { name: 'Open modal' }).click();
await expect(page.getByRole('dialog')).toBeVisible();
await page.getByRole('button', { name: 'Close' }).click();
\end{verbatim}

\subsubsection*{Navigacija kroz elemente}:

\begin{verbatim}
await page.getByRole('link', { name: 'Next' }).click();
await expect(page).toHaveURL('/next-page');
\end{verbatim}


\section{Generiranje testova}
Playwright ima mogućnost generiranja testova tako što snima klikanje miša korisnika te to prevodi u kod koji se može izvršavati.
To je vrlo jednostavan i praktičan način da čak i totalni početnici mogu krenuti s izradom automatskih testova, a kasnije s iskustvom se ti testovi mogu rafinirati i poboljšavati.
Često je i tako generirai kod dovoljno dobar za upotrebu kod jednostvanijih slučajeva, a zasigurno je dobar za brzo sastavljanje testova da se izbjegne često dosatno tipkanje djelova koda koji se neće kasnije ponovno upotrebljavati.
To uklanja potrebu za pisanje prilagođenih funkcija kao što radimo prilikom izrade projekta koji planiramo dugo vremena održavati i na taj način se može uštediti dosta vremena.
Iako, treba biti oprezan s time jer često ušteda vremena na početku vodi do puno utrošenog vremena kasnije, ali to je tema koja je izvan okvira ovog rada.
\section{Pokretanje alata za generiranje testova}

Alat za generiranje testova se pokreće putem \texttt{codegen} naredbe koja prima argument URL web stranice za koju se želi generirati testovi.
URL nije obavezan i može se pokrenuti alat bez njega, a zatim dodati URL izravno u prozoru preglednika.

Pokazati ćemo to na promjeru koji se nalazi u službenoj dokumentaciji \footnote{\url{https://playwright.dev/docs/codegen-intro\#running-codegen}} koristeći naredbu
\begin{verbatim}
npx playwright codegen demo.playwright.dev/todomvc
\end{verbatim}
\begin{figure}[!h]\begin{center}
    \includegraphics[width=1\textwidth]{"img/codegenInterface"}
    \caption{Izgled sučelja za generiranje testova}\label{img:pwCodeGen}
\end{center}\end{figure}

Na slici \ref{img:pwCodeGen} je prikazan izgled sučelja za generiranje testova koji se sastoji od nekoliko djelova:
\begin{itemize}
    \item prozor preglednika unutar kojeg se izvršava aplikacija - označen s crvenim okvirom (gore)
    \item prozor unutar kojeg se prikazuje generirani kod - označen s žutim okvirom (dolje)
    \item Lokator koji će Playwright koristiti se prikazuje kada se stavi miš preko elementa - označen sa zelenim okvirom (unutar prozora preglednika)
\end{itemize}


\section{Kontinuirana integracija i testiranje}\label{CI/CD}
Uvođenje kontinuirane integracije (CI) i kontinuirane dostave (CD) predstavlja jedan od ključnih elemenata suvremenog razvoja softvera, omogućujući timovima bržu, pouzdaniju i konzistentniju isporuku softverskih rješenja. 
U nastavku će biti opisano kako i zašto napraviti jednostavan CI/CD pipeline-a za automatizirano testiranje pomoću Playwrighta.

CI/CD pipeline je automatizirani niz koraka koji omogućuje brzu i pouzdanu isporuku aplikacija. CI/CD pipeline se sastoji od niza automatiziranih procesa koji uključuju izgradnju (build), testiranje i distribuciju (deployment) aplikacije. 
Kroz automatizaciju ovih koraka, CI/CD pipeline pomaže u smanjenju rizika od grešaka, ubrzava proces isporuke te osigurava dosljednost i kvalitetu softverskih rješenja.

% \subsection{Postavljanje Playwrighta u CI/CD pipeline}

Integracija Playwright-a u CI/CD pipeline omogućava automatizirano izvođenje testova pri svakoj promjeni koda, čime se osigurava da sve funkcionalnosti aplikacije rade ispravno prije nego što se promjene implementiraju u produkciju.

Definicija CI/CD Pipeline-a: Nakon konfiguracije repozitorija, potrebno je definirati CI/CD pipeline. To se obično radi pomoću YAML datoteka koje sadrže instrukcije za izgradnju, testiranje i distribuciju aplikacije. U nastavku je primjer osnovne konfiguracije za GitHub Actions, a može biti generirana automatski prilikom inicijalizacije projekta kao što je prikazana na slici \ref{img:ghActionPrompt}:

\begin{figure}[!h]\begin{center}
  \includegraphics[width=1\textwidth]{"img/ghActionPrompt"}
  \caption{Upit za kreiranje GitHub Action workflowa prilikom inicijalizacije projekta}\label{img:ghActionPrompt}
\end{center}\end{figure}

\begin{verbatim}
name: Playwright Tests
on:
  push:
    branches: [ main, master ]
  pull_request:
    branches: [ main, master ]

jobs:
  test:
    timeout-minutes: 60
    runs-on: ubuntu-latest

    steps:
    - uses: actions/checkout@v4
    - uses: actions/setup-node@v4
      with:
        node-version: lts/*

    - name: Install dependencies
      run: npm ci

    - name: Install Playwright Browsers
      run: npx playwright install --with-deps

    - name: Run Playwright tests
      run: npx playwright test

    - uses: actions/upload-artifact@v4
      if: always()
      with:
        name: playwright-report
        path: playwright-report/
        retention-days: 30
\end{verbatim}
Ova konfiguracija definira akcije koje će se pokrenuti pri svakom push-u ili pull request-u na glavnu granu repozitorija. Akcije uključuju preuzimanje koda, postavljanje Node.js okruženja, instalaciju zavisnosti i pokretanje Playwright testova.

Nakon definiranja CI/CD pipeline-a, svaki push ili pull request pokreće automatiziranu izgradnju i testiranje aplikacije. 
Playwright testovi se izvršavaju unutar pipeline-a, čime se osigurava da sve promjene koda ne narušavaju postojeću funkcionalnost.

Posljednji korak CI/CD pipeline-a je distribucija aplikacije. 
Ako svi testovi prođu uspješno, aplikacija se automatski distribuira na produkcijsko okruženje. 
Ovaj korak može uključivati različite metode distribucije, kao što su deployment na cloud platforme, generiranje Docker datoteke (image) ili distribucija na serverske klastere.

\subsection*{Prednosti CI/CD Pipelinea za Playwright}
Implementacija CI/CD pipelinea za Playwright donosi brojne prednosti:
\begin{itemize}
    \item Automatizacija: CI/CD pipeline automatizira proces testiranja i distribucije, čime se smanjuje potreba za ručnim intervencijama i povećava produktivnost tima.
    \item Konzistentnost: Automatizirano testiranje osigurava dosljednost i kvalitetu aplikacije, jer se testovi izvršavaju pri svakoj promjeni koda.
    \item Brža isporuka: CI/CD pipeline ubrzava proces isporuke, omogućujući brže uvođenje novih funkcionalnosti i popravaka grešaka.
    \item Rano otkrivanje problema: Redovito izvršavanje testova omogućava rano otkrivanje problema, čime se smanjuje rizik od grešaka u produkcijskom okruženju.
\end{itemize}

\subsection*{Implementacija kontinuiranog testiranja unutar Azure DevOps okruženja}

Konfiguracija u YAML formatu:
\begin{verbatim}
  steps:
- script: 'npx playwright test tests/postDeployPipelineQA.test.ts --workers 1'
  workingDirectory: '$(System.DefaultWorkingDirectory)/Playwright_Source/postDeployTests'
  displayName: 'Run The Test'
\end{verbatim}


\begin{figure}[!h]\begin{center}
  \includegraphics[width=1\textwidth]{"img/CI_pipelinePreview"}
  \caption{Izgled Azure Pipelinea sa zasebnim koracima}\label{img:CI_pipelinePreview}
\end{center}\end{figure}

\begin{figure}[!h]\begin{center}
  \includegraphics[width=1\textwidth]{"img/CI_pipelineDetails"}
  \caption{Detalji koraka za testiranje - posljednji u nizu unutar pipelinea}\label{img:CI_pipelineDetails}
\end{center}\end{figure}

\begin{figure}[!h]\begin{center}
  \includegraphics[width=1\textwidth]{"img/CI_pipeline"}
  \caption{Detalji koraka za testiranje - posljednji u nizu unutar pipelinea}\label{img:CI_pipeline}
\end{center}\end{figure}

\begin{figure}[!h]\begin{center}
  \includegraphics[width=1\textwidth]{"img/CI_testConfig"}
  \caption{Detalji koraka za testiranje - posljednji u nizu unutar pipelinea}\label{img:CI_testConfig}
\end{center}\end{figure}