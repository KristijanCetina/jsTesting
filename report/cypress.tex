\chapter{Cypress}\label{ch_cypress}
Cypress se reklamira kao moderan alat za testiranje fronend aplikacija nove generacije \cite{cypressDocsPage}.
Često se uspoređuje s drugim popularnim alatom - Seleniumom.
Međutim, Cypress i Selenium se u mnogičemu razlikuju, a glavne razlike se tiču samog pristupa testiranju i samim time drukčije arhitekture što omogućava Cypressu kompetitivne prednosti.
Više o tome u nastavku.

Cypress omogućava pisanje i izvršavanje raznih tipova testova kao što su:
\begin{itemize}
    \item testovi na krajnjim točkama (\textit{end-to-end test, e2e test})
    \item Integracijski testovi
    \item Unit testovi
\end{itemize}
U kratko, Cypress može testirati sve što se izvršava u browseru.

\section{Opis i pregled paketa}
Glavne značajke Cypress alata su:
\subsection*{Putovanje kroz vrijeme (\textit{Time Travel})}
Cypress sprema snimke stanja kako prolazi kroz testove kako bi kasnije mogli pregledati što se točno događalo tokom izvršavanja. 
To nam omogućava da točno vidimo zašto neki test nije prošao i u kojem je stanju aplikacija bila u tom trenutku.


\section{Instalacija}

\section{Osnovni test}

\section{Kontinuirana integracija - CI}
