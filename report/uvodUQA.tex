\chapter{Uvod u testiranje programskog rješenje i osiguranje kvalitete}\label{uvodQA}

Testianje softwareskog proizvoda se provodi u cilju osiguravanja kvalitete samog proizvoda.
Svaka faza razvoja ima svoje specifične zahtjeve i načine testiranja, a njihov pregled je dan na slici \ref{img:testingInScrum}.
Pojedine funkcije u kodu se pokrivaju unit testovima koji se brinu da pojedine komponente rade ono što su namjenjene na najnižoj razini. To su testovi koji se u pravilu izvršavaju vrlo često prilikom pisanja koda kako bi se osiguralo da promjena neke funkcije ili komponente nije poremetila njen rezultat.


\begin{figure}[!h]\begin{center}
\includegraphics[width=0.8\textwidth]{"img/testPhasesScrum"}
\caption{Proces testiranja u Scrum metodologiji rada}\label{img:testingInScrum}
\end{center}\end{figure}

\section{Proces testiranja}
Svaki proces testiranja započinje sa izradom plana testiranja unutar kojeg se definira šte se testira na koji način te koji su uvjeti da se zadani test smatra uspješnim (\textit{engl. Acceptance critera}).

Tokom samog razvijanja nove funkcionalnosti u softwareskom prozvodu često se izvršava ručno testiranje kako bi se utvrdilo da li proces razvoja ide u zadanom smjenu i u konačnici da se radi ono što je dogovoreno bilo interno s timom ili čak i sa klijentom.

Nakon što se završi sam razvoj nove funkcionalnosti onda se radi i završno funkcionalno testiranje i izrada automatskih testova koji će se u budućnosti izvršavati automatski u zadanom intervalu kako bi se osiguralo da uvođenje nove funkcionalnosti ne uvode nove pogreške na već ispravnim funkcionalnostima.
Poznato kao i regresijsko testiranje.

Bitno je testirati realne scenarije koji se očekuju da moraju zadovoljiti,  kao i one scenarije od kojih se očekuje da nesmiju proći test.
To se radi u svrhu potvrde da test zaista radi ono što je namijenjen,  a ne da imamo propust u samom testu koji uvijek vraća pozitivan rezultat ili da testirana funkcija nema implementiranu validaciju ulaznih parametara koji onda mogu izazvati nepoželjno ponašanje programa.

\section{Ciljevi testiranja}
Ciljevi testiranja moraju zadovoljavati nekoliko kriterija, a to su:
\begin{itemize}
\item Specifičnost
\item Mjerljivost
\item Ostvarljiv
\item Realističan
\item Vremenski ograničen
\end{itemize}

Neki od ciljeva testiranja su: \cite{quadri2010software}
\subsection*{Verifikacija i validacija}
Cilja testiranja nije samo pronaći pogreške u kodu ili dizajnu.  
Cilj je verificirati da software zaista radi ono što je namjenjen i kako je zamišljen.
Jedan od rezultata testiranja je i izvještaj (\textit{test report}).

\subsection*{Prioretiziranje pokrivenosti}
U idealnom svijetu sa neograničenim resursima svaki dio koda i funkcionalnosti bi bio pokriven testovima, ali nažalost to nije moguće.
Zato je bitno pravilno odrediti što je prioritet te što će se pokriti testovima. 
U pravilu su to one funkcionalnosti i značajke koje nisu vidljive na prvi pogled čim se otvori program jer su takvi problemi lako uočljivi svakome.
Isto tako, beskonačni testovi uzimaju mnogo dragocjenog vremena pa je i u tom pogledu bitno odrediti što se treba testirati.

\subsection*{Sljedivost}
Dokumentiranje testova kada se nešto i kako testiralo je bitno kako bi se u slučaju pojave problema moglo odrediti kada je i koja promjena uzrokovala neželjeno ponašanje proizvoda.
To je posebno bitno u odreženim kategorijama softwarea kao npr. u financijskom poslovanju gdje se može tražiti dodatna odgovornost samog proizvoda.

\section{Uloga testera u timu}
Glavna uloga testera, kao i samog procesa testiranja, je dodatna sigurnosna mreža koja je samo zadnja karika u lancu procesa testiranja i osiguranja kvalitete proizvoda.
Dok su programeri (\textit{developeri, engl. developers}) zaduženi za pisanje unit testova,  testeri su zaduženi za pisanje i izvršavanje funkcionalnih testova.
Testeri također pomažu product owneru sa izradom kriterija za uspješno prihvačanje rezultata testova \cite{mundra2013practical}.
Bitno je napomenuti kako su i developeri i testeri dio istog tima te  kao tim imaju zajednički cilj - isporuka najkvalitetnijeg proizvoda moguće unutar zadanih parametara.
