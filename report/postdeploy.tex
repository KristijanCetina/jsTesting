\chapter{Testiranje nakon nadogradnje na novu verziju}\label{postdeploy}

Jedna od ključnih aktivnosti svakog razvojnog tima je redovita nadogradnja softvera na novu verziju.
U našem poduzeću, uobičajena praksa je implementacija nove verzije softvera jednom mjesečno za glavnog klijenta.
S obzirom na to da klijent ima postrojenja na šest kontinenata, uključujući dodatna postrojenja smještena u međunarodnim vodama svih svjetskih oceana, proces nadogradnje podijeljen je po regijama.

Te regije su:
\begin{itemize}
    \item Americas (Sjeverna i Južna Amerika),
    \item EMEA (Europa, Bliski istok i Afrika),
    \item APAC (Azija i Pacifik).
\end{itemize}

Svaka od ovih regija obuhvaća više od 15 postrojenja koja koriste zasebne poslužitelje, što znači da se nadogradnja mora provesti na više od 15 lokacija unutar svake regije.
Već iz ove strukture vidljivo je da proces nadogradnje nosi sa sobom značajan rizik od potencijalnih problema i zastoja koji se mogu pojaviti tijekom implementacije.
Stoga je od ključne važnosti osigurati da proces nadogradnje bude što optimalniji, uz minimalne prekide u radu i bez nepotrebnih zastoja.

\section{Postojeći način testiranja}\label{current_postdeploy}
rije uvođenja automatskih testova, standardna procedura uključivala je detaljno testiranje AM regije koje je provodilo 3 do 4 testera, odnosno inženjera za kontrolu kvalitete (QA).
Svaki tester bi provjeravao 3 do 4 postrojenja (site).
U prosjeku, za provjeru jednog postrojenja bilo je potrebno između 20 i 30 minuta kako bi se osiguralo da su sve promjene ispravno primijenjene te da nisu izazvale neželjene nuspojave ili prestanak rada postojećih funkcionalnosti.

S obzirom na velik broj funkcionalnosti i ograničeno vrijeme, bilo je nužno selektivno pristupiti odabiru onih funkcionalnosti koje će se testirati, pri čemu je ključni kriterij bila kritičnost pojedine funkcionalnosti.
Cijeli proces trajao je između 1,5 i 2 sata po članu testnog tima, dok je ukupni trošak procijenjen na 6 do 7 radnih čovjek-sati.
Jednostavnom množenjem s cijenom rada dolazi se do ukupnog troška testiranja nakon nadogradnje.

Nakon što bi se verificiralo da je novi paket programa ispravan, te u slučaju da klijenti nisu prijavili ozbiljne probleme koji bi onemogućili daljnji rad (\emph{showstopper}), pristupilo bi se nadogradnji sljedeće regije, Azije i Pacifika.

U tom bi se slučaju postupak ponovio s mogućim izmjenama, poput smanjenja broja članova testnog tima za jednog, kako bi se izbjegli prekovremeni sati.
Ova redukcija testiranja značila je da se provjeravao samo smanjeni set funkcionalnosti, što je zahtijevalo maksimalno 20 minuta po postrojenju, odnosno ukupno oko 5 čovjek-sati po regiji.
Iako se na taj način smanjio trošak, on je i dalje bio značajan, pogotovo jer se rad na AP regiji djelomično odvijao izvan redovnog radnog vremena, što je povećavalo trošak prekovremenog rada.
Ovaj problem bio je posebno izražen za EU regiju, gdje se radovi obavljaju potpuno izvan radnog vremena, pa je sav rad bio prekovremeni.

Uz sve navedeno, dodatni problem ručnog testiranja bila je točnost i preciznost, pri čemu su različiti testeri provodili testiranja na različitim razinama kvalitete.
To je razumljivo, s obzirom na to da se radi o ljudima, a ne o strojevima.
Stoga smo odlučili taj posao, u što većoj mjeri, prepustiti strojevima.

Iz navedenog je jasno kako je bilo nužno unaprijediti proces testiranja nadogradnje, prvenstveno radi povećanja kvalitete rada.

\section{Automatsko testiranje procesa nadogradnje}

Glavni cilj prelaska na automatski način testiranja je bio povećati preciznost testova i osigurati da se ne izostavi niti jedan korak koji se provjeravao pri ručnom testiranju.

Drugi cilj je bio smanjiti vrijeme potrebno za testiranje s prethodnih 30 minuta na ispod 5 minuta.

Dodatan cilj je bio omogućiti daljnji rast broja klijenata i instalacija koja se mogu nadograditi istovremeno, a bez da se mora uložiti značajno više vremena.
Idealno smo htjeli postići $\mathcal{O}(\log n)$ rast potrošenog vremena za svakog novog klijenta.

Nakon što je provedena nadogradnja s automatskim testiranjem potvrđeno je da su glavni ciljevi ostvareni te da će se nastaviti s implementacijom primijenjene tehnologije i za ostale potrebe.
Konkretno, vrijeme testa po postrojenju je iznosilo od 2-4 minuta, ovisno o performansama servera i količini podataka koje klijent ima za razne izvještaje.
Na slici \ref{img:fullTestPass} je prikazan izvještaj koji Playwright generira nakon završenog testiranja.
\begin{figure}[!h]\begin{center}
    \includegraphics[width=0.8\textwidth]{"img/fullTestPass"}
    \caption{Izvještaj nakon završenog automatskog testiranja}\label{img:fullTestPass}
\end{center}\end{figure}
 
Niti jedan test nije pokazivao znakove problema, ako ih zaista nije bilo (\emph{false-negative} testovi).
Svi uočeni problemi nakon isporuke nove verzije su bili riješeni u rekordnom roku, a tome je pridonijelo vrlo kratko vrijeme do otkrivanja problema.

Kao dodatan benefit je primijećeno da se sada testeri mogu više fokusirati na specifične rubne slučajeve koje je potrebno verificirati te tako dodatno smanjiti broj prijavljenih nedostataka od strane korisnika te posljedično povisiti kvalitetu isporučenog proizvoda.
