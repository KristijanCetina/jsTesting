\chapter{Testiranje nakon nadogradnje na novu verziju}\label{postdeploy}

Jedna od čestih aktivnosti svakog razvojnog tima je nadogradnja programa na novu verziju.
U poduzeću je praksa da se pušta u produkcijski rad nova verzija programa jednom mjesečno za glavnog klijenta. 
Kako klijent ima postrojenja na 6 kontinenta uz dodatna postrojenja koja se nalaze u međunarodnim vodama svih svjetskih oceana proces nadogreadnje je podijeljen po regijama.
To su Americas (Sjeverja i Južna Amerika), EU (Europa i Bliski istok + Afrika) te AP (Azija i Pacifik).
Svaka od regije ima više od 15 postrojenja koja koriste zasebne servere što znači da se nadogradnja vrši na više od 15 lokacija po regiji.
Već iz ovoga je izvjesno kako je to puno potencijalnih problema i zastoja koja se mogu pojaviti te treba osigurati da je proces nadogranje čim je više optimalan bez nepotrebnih zastoja i prekida u radu.

\section{Postojeći način testiranja}
Do uvođenja automstkih testova standardna procedura se sastojala od toga da AM regiju detaljno testira 3-4 testera (inženjera za kontrolu kvalitete - QA) koji bi svaki od njih provjerio 3 do 4 postrojenja (site).
U prosjeku, za provjeriti jedno postrojenje je trebalo 20 do 30 minuta kako bi se osiguralo da su sve promjene aplicirane ispravno te da nisu izazvale neželjene nuspojave i prestanak rada postojećih funkcionalnosti.
Naravno, zbog velikog broja funkcionalnosti te nedostatka vremena da se svaka od njih ručno provjeri bili smo vrlo izbirljivi što će se testirati s obzirom na kritičnost neke funkcionalnosti.
Cijeli taj proces bi trajao oko 1:30 do 2 sata po članu test tima, a sve ukupno bi se trošak procijenio na 6-7 radnih čovjek-sati.
Ako to pomnoćimo s cijenom rada lako se dolazi do ukupnog troška testiranja nakon nadogradnje.

Nakon što se verificiralo da je novi paket programa ispravan, te ako klijenti nisu prijavili ozbiljne probleme koji ih spriječavaju u radu putem sustava za prijavu kvara (\emph{showstopper}), moglo se pristupiti nadogradnji sljedeće regije Asia Pacific.

Tada bi se ponovila opet ista procedura, uz eventualne izmjene da bi bio 1 član testinog tima manje kako bi se smanjili prekovremeni sati jer nije bilo potrebe za provjerom svih funkcionalnosti, već je bilo prihvatljivo da se provjeri i reducirani set za koji bi trebalo maksimalno 20 minuta po postrojenju što bi ukupno iznosilo oko 5 čovjek-sati po regiji.
Iako je to manji trošak, i dalje je značajni trošak. 
Čim više što se je termin za radove na AP regiji djelomično izvan rednovnog radnog vremena poduzeća tako da treba uračanti i trošak prekovremenog rada.
To je posebno izraženo za EU regiju kod koje pak termin za radove je u potpunosti izvan redovnog radnog vremena pa je u tom slučaju sav rad je prekovremeni rad.