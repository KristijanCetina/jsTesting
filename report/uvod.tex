\chapter{Uvod i opis zadatka}\label{OpisIOgranicenja}
U dinamičnom svijetu razvoja softvera, automatizacija testiranja igra ključnu ulogu u osiguravanju kvalitete proizvoda.
End-to-end (E2E) testiranje je pristup koji omogućuje simulaciju stvarnog korisničkog iskustva kroz cijeli sustav, od početka do kraja.
Cilj ovih testova je potvrditi da svi dijelovi sustava funkcioniraju zajedno na predviđeni način.
Sa sve većom složenošću softverskih sustava i potrebom za bržim izdanjima, automatsko E2E testiranje postaje sve relevantnije. 
Međutim, ovo testiranje donosi sa sobom niz izazova koje je potrebno adresirati kako bi bilo učinkovito i korisno.


\section*{Opis i definicija problema}
Automatsko end-to-end testiranje softvera obuhvaća proces kreiranja, izvršavanja i održavanja testova koji provjeravaju funkcionalnost aplikacije kao cjeline.
Problem koji se istražuje u ovom radu može se definirati kroz sljedeće ključne aspekte:

Složenost Kreiranja Testova: Kreiranje automatskih E2E testova zahtijeva duboko razumijevanje svih dijelova sustava i njihovih međusobnih interakcija.
Ovo može biti izuzetno složeno u velikim i distribuiranim sustavima.

Održavanje Testova: S obzirom na česte promjene u kodu i sistemskim zahtjevima, automatski E2E testovi zahtijevaju redovno održavanje kako bi ostali relevantni.
Svaka promjena može potencijalno zahtijevati modifikaciju ili kreiranje novih testova.

Izvršavanje Testova: Automatski E2E testovi često traju duže od drugih vrsta testova (kao što su unit testovi ili integracijski testovi) zbog svoje prirode koja obuhvaća cijeli sustav. 
Ovo može rezultirati dugim vremenom izvršavanja i problemima s performansama.

Pouzdanost Testova: Testovi moraju biti pouzdani, tj. rezultati testiranja moraju biti točni i konzistentni. 
Lažno pozitivni ili negativni rezultati mogu dovesti do gubitka povjerenja u testove i dodatnih troškova.

Integracija sa CI/CD Procesima: Automatsko E2E testiranje mora biti integrirano s kontinuiranim integracijskim i kontinuiranim isporučnim (CI/CD) procesima kako bi podržalo agilne prakse razvoja softvera. Ovo zahtijeva visok nivo automatizacije i orkestracije.


\section*{Struktura rada}
Struktura ovoga rada podjeljena je u logičke cjeline.
Nakon uvoda i objašnjavanja rada, u poglavlju \ref{uvodQA} - \nameref{uvodQA} objašnjen je proces testiranje i osiguranja kvalitete programskog rješenja.

Poglavlje \ref{ch_playwright} - \nameref{ch_playwright} detaljnije objašnjava korištenu biblioteku kao i njezino korištenje.

Poglavlje \ref{postdeploy} - \nameref{postdeploy} objašnjava proces testiranja nakon objave nove verzije kao i problem koji se pokušava rješiti ovim radom.

Poglavlje \ref{ch:implementacija} - \nameref{ch:implementacija} pruža detaljniji uvid u samo rješenje koje je implementirano unutar kompanije u kojoj autor trenutno radi.

Kompletan Git repozitorij ovog rada javno je dostupan na \url{https://github.com/KristijanCetina/jsTesting}