\chapter{Uvod i opis zadatka}\label{OpisIOgranicenja}
Tema ovog rada proizašla je iz autorove želje za proučavanjem tematike te kao gorljivim poklonikom metode učenja kroz praktičan rad i primjenu stečenog znanja i iskustva na rješavanje realnog problema.


\section{Opis i definicija problema}
gdfg

\section{Cilj i svrha rada}
yeryy

\section{Hipoteza rada}
Hipoteza ovog rada je da promjenom primjerenih metoda testiranja programskog proizvoda može se značajno smanjati količina grešaka (\textit{bugova, engl. bugs)} u finalnom proizvodu koji se isporučuje krajnjem korisniku te ostvariti uštede u resursima za njihovo ispravljanje.

\section{Metode rada}
Tijekom izrade ovoga rada korištene su različite znanstveno-istraživačke metode od kojih je svaka najprikladnija postavljenom izazovu, a one su:
\begin{itemize}
\item Istraživačka metoda - za stjecanje uvida u zadane okvire zadatka
\item Metoda logičke analize i sinteze - za prikupljanje podataka iz literature
\item Deskriptivna metoda - za izradu uvodnog i završnog dijela projektnog zadatka
\item Eksperimentalna metoda - u potrazi za optimalnim rješenjima za zadani dio problema
\end{itemize}

\section{Struktura rada}
Struktura ovoga rada podjeljena je u logičke cjeline.
Nakon uvoda i objašnjavanja rada, u poglavlju 

Kompletan Git repozitorij ovog rada javno je dostupan na \url{https://github.com/KristijanCetina/jsTesting}