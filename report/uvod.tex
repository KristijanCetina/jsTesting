\chapter{Uvod}\label{OpisIOgranicenja}
U kontekstu sve bržeg razvoja softverskih rješenja i povećanih zahtjeva za kvalitetom, automatizacija testiranja se etablirala kao nezaobilazni dio razvojnog procesa. Među različitim vrstama testiranja, end-to-end (E2E) testiranje zauzima posebno mjesto, simulirajući realistične scenarije korištenja kako bi se osiguralo da softverski sustav funkcionira kao koherentna cjelina.

E2E testiranje predstavlja sveobuhvatan pristup provjeri funkcionalnosti aplikacije od početka do kraja, uključujući interakciju s različitim komponentama, bazama podataka, vanjskim sustavima i korisničkim sučeljima. Glavni cilj ovakvog testiranja je potvrditi da su svi dijelovi sustava integrirani i da zajedno ispunjavaju postavljene funkcionalne zahtjeve.

S obzirom na sve veću kompleksnost modernih softverskih sustava, koji često uključuju mnoštvo međusobno povezanih komponenti i usluga, ručno provođenje E2E testova postaje sve zahtjevnije i sklono pogreškama. Automatizacija ovih testova omogućuje značajnu uštedu vremena i resursa, te osigurava dosljednost i ponovljivost testiranja.
x
Međutim, implementacija automatiziranih E2E testova nije bez izazova. Neki od najčešćih problema uključuju:
\begin{itemize}
    \item Krhkost testova: Promjene u korisničkom sučelju ili podacima mogu dovesti do čestih prekida testova, što zahtijeva kontinuirano održavanje.
    \item Složenost postavke: Konfiguracija okruženja za E2E testiranje može biti kompleksna, posebno za velike i distribuirane sustave.
    \item Vrijeme izvršavanja: E2E testovi su često dugotrajni, što može usporiti razvojni proces.
    \item Odabir pravih alata: Izbor odgovarajućih alata za automatizaciju E2E testiranja ovisi o specifičnim potrebama projekta.
\end{itemize}

Usprkos navedenim izazovima, prednosti automatiziranog E2E testiranja su neosporne.
Ovakav pristup omogućuje rano otkrivanje i otklanjanje grešaka, poboljšava kvalitetu softvera, smanjuje troškove održavanja i ubrzava proces izdanja.


\section*{Opis i definicija problema}
Automatizirano end-to-end (E2E) testiranje predstavlja integralni dio suvremenog razvoja softvera, omogućavajući sveobuhvatnu provjeru funkcionalnosti aplikacije od početka do kraja.
Međutim, implementacija i održavanje učinkovitih E2E testova suočava se s nizom izazova koji ograničavaju njihovu sveobuhvatnu primjenu.

Složenost dizajna testova predstavlja jedan od ključnih problema.
Kreiranje robusnih E2E testova zahtijeva duboko razumijevanje kompleksnih interakcija unutar sustava, što se posebno ističe u velikim i distribuiranim arhitekturama.
Svaka promjena u sustavu može utjecati na višestruke testove, zahtijevajući njihovu reviziju ili ponovno pisanje.

Održivost testova je još jedan izazov koji se javlja zbog dinamične prirode razvoja softvera.
Česte promjene u kodu, korisničkom sučelju i podacima mogu dovesti do zastarjelosti postojećih testova, što zahtijeva kontinuirana ulaganja u njihovo održavanje.

Vrijeme izvršavanja predstavlja značajnu prepreku za široku primjenu E2E testova.
Zbog sveobuhvatne prirode, ovi testovi obično zahtijevaju značajno više vremena za izvršavanje u usporedbi s drugim vrstama testova (npr. unit testovima).
Dugo trajanje izvršavanja može usporiti razvojni proces i otežati brzu povratnu informaciju.

Pouzdanost rezultata je ključna za uspjeh automatiziranih testova.
Lažno pozitivni ili negativni rezultati mogu dovesti do gubitka povjerenja u testnu infrastrukturu i uzrokovati nepotrebne troškove.
Faktori koji utječu na pouzdanost uključuju stabilnost testnog okruženja, kvalitetu testnih podataka i robusnost same testne logike.

Integracija s CI/CD procesima je još jedan aspekt koji treba pažljivo razmotriti.
Uspješna implementacija automatiziranih E2E testova zahtijeva njihovu tijesnu integraciju s kontinuiranim integracijskim i isporučnim (CI/CD) procesima.
Ovo podrazumijeva visok stupanj automatizacije, orkestracije i usklađenosti s postojećim razvojnim praksama.

\section*{Struktura rada}
Struktura ovoga rada podijeljena je u logičke cjeline.
Nakon uvoda i objašnjavanja rada, u poglavlju \ref{uvodQA} - \nameref{uvodQA} objašnjen je proces testiranje i osiguranja kvalitete programskog rješenja.

Poglavlje \ref{ch_playwright} - \nameref{ch_playwright} detaljnije objašnjava korištenu biblioteku kao i njezino korištenje.

Poglavlje \ref{postdeploy} - \nameref{postdeploy} objašnjava proces testiranja nakon objave nove verzije kao i problem koji se pokušava riješiti ovim radom.

Poglavlje \ref{ch:implementacija} - \nameref{ch:implementacija} pruža detaljniji uvid u samo rješenje koje je implementirano unutar kompanije u kojoj autor trenutno radi.

Poglavlje \ref{ch:Zakljucak} - \nameref{ch:Zakljucak} je završni zaključak rada.

Kompletan Git repozitorij ovog rada javno je dostupan na \url{https://github.com/KristijanCetina/jsTesting}