\section*{Sažetak}\label{sazetak_hr}
\addcontentsline{toc}{chapter}{\nameref{sazetak_hr}}
Često se kaže da je testiranje softvera jednako važno kao i samo kodiranje. 
Kako se kompleksnost aplikacija povećava, tako raste i vjerojatnost pojave pogrešaka koje mogu negativno utjecati na korisničko iskustvo. 
Da bismo to spriječili, koristimo razne tehnike testiranja, a jedna od njih je testiranje klijentskih komponenti.

Ovaj rad se fokusira na korištenje Microsoftovog alata Playwright za testiranje klijentskih komponenti. 
Playwright je popularan izbor za end-to-end testiranje modernih web aplikacija jer omogućuje pouzdano testiranje na različitim preglednicima i platformama. 
Njegove glavne prednosti uključuju podršku za najnovije web tehnologije, jednostavnu sintaksu i mogućnost pisanja robusnih i održavanih testova.

\subsection*{Ključne riječi}\label{kw_hr}
\addcontentsline{toc}{section}{\nameref{kw_hr}}
\textit{Playwright, JavaScript, open-source}

\section*{Sommario}\label{sazetak_it}
\addcontentsline{toc}{chapter}{\nameref{sazetak_it}}
Si dice spesso che testare il software sia importante quanto la codifica stessa.
Con l'aumento della complessità delle applicazioni, aumenta anche la probabilità di errori che possono influire negativamente sull'esperienza utente.
Per prevenirlo, utilizziamo diverse tecniche di test, una delle quali è il test dei componenti client-side.

Questo documento si concentra sull'utilizzo di Playwright di Microsoft per testare i componenti client-side.
Playwright è una scelta popolare per i test end-to-end delle moderne applicazioni web in quanto consente test affidabili su diversi browser e piattaforme.
I suoi principali vantaggi includono il supporto per le ultime tecnologie web, una sintassi semplice e la capacità di scrivere test robusti e manutenibili.

\subsection*{Parole chiave:}\label{kw_it}
\addcontentsline{toc}{section}{\nameref{kw_it}}
\textit{Playwright, JavaScript, open-source}

\section*{Abstract}\label{sazetak_en}
\addcontentsline{toc}{chapter}{\nameref{sazetak_en}}
It's often said that testing software is equally important as coding itself.

 As application complexity grows, so does the likelihood of errors that can negatively impact the user experience.
 To prevent this, we use various testing techniques, one of which is testing client-side components.

This paper focuses on using Microsoft's Playwright for testing client-side components.
Playwright is a popular choice for end-to-end testing of modern web applications as it enables reliable testing across different browsers and platforms.
Its main advantages include support for the latest web technologies, simple syntax, and the ability to write robust and maintainable tests.

\subsection*{Keywords:}\label{kw_en}
\addcontentsline{toc}{section}{\nameref{kw_en}}
\textit{Playwright, JavaScript, open-source}