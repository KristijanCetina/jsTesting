\section*{Sažetak}\label{sazetak_hr}
\addcontentsline{toc}{chapter}{\nameref{sazetak_hr}}
Kako bi se osigurala isporuka kvalitetnog programskog rješenja, testiranje softvera je jednako važno kao i samo kodiranje.
Kako se kompleksnost aplikacija povećava, raste i vjerojatnost pojave pogrešaka koje mogu negativno utjecati na korisničko iskustvo.
Kako bismo to spriječili, koristimo razne tehnike testiranja, a jedna od njih je testiranje klijentskih komponenti.

Ovaj rad se fokusira na korištenje Microsoftovog alata Playwright za testiranje klijentskih komponenti.
Playwright je popularan izbor za end-to-end testiranje modernih web aplikacija jer omogućuje pouzdano testiranje na različitim preglednicima i platformama.
Njegove glavne prednosti uključuju podršku za najnovije web tehnologije, jednostavnu sintaksu i mogućnost pisanja robusnih i održivih testova.
\subsection*{Ključne riječi}\label{kw_hr}
\addcontentsline{toc}{section}{\nameref{kw_hr}}
\textit{Playwright, JavaScript, open-source}

\section*{Sommario}\label{sazetak_it}
\addcontentsline{toc}{chapter}{\nameref{sazetak_it}}
Per garantire la consegna di una soluzione software di qualità, il test del software è importante tanto quanto la programmazione stessa.
Con l'aumentare della complessità delle applicazioni, cresce anche la probabilità di errori che possono influire negativamente sull'esperienza utente.
Per evitare ciò, utilizziamo diverse tecniche di test, una delle quali è il test delle componenti client-side.

Questo lavoro si concentra sull'uso dello strumento Playwright di Microsoft per testare le componenti client-side.
Playwright è una scelta popolare per il testing end-to-end delle applicazioni web moderne perché consente di testare in modo affidabile su diversi browser e piattaforme.
I suoi principali vantaggi includono il supporto per le tecnologie web più recenti, una sintassi semplice e la capacità di scrivere test robusti e sostenibili.

\subsection*{Parole chiave:}\label{kw_it}
\addcontentsline{toc}{section}{\nameref{kw_it}}
\textit{Playwright, JavaScript, open-source}

\section*{Abstract}\label{sazetak_en}
\addcontentsline{toc}{chapter}{\nameref{sazetak_en}}
To ensure the delivery of a quality software solution, software testing is just as important as coding itself.
As the complexity of applications increases, so does the likelihood of errors that can negatively impact the user experience.
To prevent this, we use various testing techniques, one of which is client-side component testing.

This paper focuses on using Microsoft's Playwright tool for testing client-side components.
Playwright is a popular choice for end-to-end testing of modern web applications because it allows reliable testing across different browsers and platforms.
Its main advantages include support for the latest web technologies, simple syntax, and the ability to write robust and maintainable tests.

\subsection*{Keywords:}\label{kw_en}
\addcontentsline{toc}{section}{\nameref{kw_en}}
\textit{Playwright, JavaScript, open-source}