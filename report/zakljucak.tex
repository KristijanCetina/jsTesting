\chapter{Zaključak}\label{ch:Zakljucak}
U ovom radu istražena je važnost testiranja klijentskih komponenti u procesu razvoja softvera, s posebnim naglaskom na korištenje Microsoftovog alata Playwright. 
Kako se složenost modernih web aplikacija povećava, pouzdano i efikasno testiranje postaje ključno za održavanje kvalitete i funkcionalnosti proizvoda. 
Playwright se istaknuo kao moćan alat za end-to-end testiranje zbog svoje podrške za najnovije web tehnologije, jednostavne sintakse i mogućnosti provođenja robusnih i održavanih testova na različitim preglednicima i platformama.

Rezultati ovog istraživanja pokazuju da korištenje Playwrighta može značajno smanjiti broj grešaka koje prolaze nezamijećene do korisničkog iskustva, čime se poboljšava ukupna kvaliteta proizvoda. 
Nadalje, implementacija Playwrighta omogućuje timovima za razvoj softvera da brže i efikasnije identificiraju i otklone pogreške, što rezultira smanjenjem troškova i vremena potrebnog za održavanje i nadogradnju aplikacija.

U zaključku, integracija Playwrighta u proces razvoja web aplikacija predstavlja vrijednu investiciju koja može donijeti dugoročne benefite u smislu pouzdanosti, kvalitete i zadovoljstva korisnika.
Buduća istraživanja mogu se fokusirati na dodatne tehnike i alate za testiranje kako bi se još više poboljšao proces osiguranja kvalitete u razvoju softvera.